% Created 2021-12-17 Fri 15:34
% Intended LaTeX compiler: pdflatex
\documentclass[11pt]{article}
\usepackage[utf8]{inputenc}
\usepackage[T1]{fontenc}
\usepackage{graphicx}
\usepackage{grffile}
\usepackage{longtable}
\usepackage{wrapfig}
\usepackage{rotating}
\usepackage[normalem]{ulem}
\usepackage{amsmath}
\usepackage{textcomp}
\usepackage{amssymb}
\usepackage{capt-of}
\usepackage{hyperref}
\author{Ali Kadum Hassan, Frederik Henriques Altmann, Gustav Mark-Hansen}
\date{\today}
\title{Hold 09 mach: Opg3}
\hypersetup{
 pdfauthor={Ali Kadum Hassan, Frederik Henriques Altmann, Gustav Mark-Hansen},
 pdftitle={Hold 09 mach: Opg3},
 pdfkeywords={},
 pdfsubject={},
 pdfcreator={Emacs 27.2 (Org mode 9.5)}, 
 pdflang={English}}
\begin{document}

\maketitle

\section*{1}
\label{sec:org05bc15b}

\subsection*{a}
\label{sec:orgd8c1f92}
Chancen for at få X kast der lander på krone i træk, er:
\((\frac{1}{2})^X\)

Skrevet i R:
\begin{verbatim}
pmf = \(throws) 0.5^throws
\end{verbatim}
\subsection*{b}
\label{sec:orgc62789a}
Profiten \(f(X)\) er givet ved \(f(a, X) = a2^X - a\)

Skrevet i R:
\begin{verbatim}
profit = \(buyin, throws) buyin * 2^(throws) - buyin
\end{verbatim}
\subsection*{c}
\label{sec:org6689ae5}
\begin{align}
E(X) = \sum_{n=1}^\infty |f(b,n)|p(n) = \sum_{n=1}^\infty b \\
|f(x)|p(x) = b2^{x+1}0.5^x = b \\
\sum_{n=1}^\infty |f(b,n)|p(n) = \sum_{n=1}^\infty b = \infty
\end{align}

Da \(E(X)\) divergerer mod positiv uendelig har spillet ikke noget forventet udfald.
\section*{2}
\label{sec:org2284d3a}
\subsection*{a}
\label{sec:orgf15ed8c}
Da hele pmf \(p\), bortset fra \(p(x=-1,y=3)\) er kendt kan \(z\) beregnes ved at isolere.

\begin{center}
\begin{tabular}{lrrl}
p(x, y) & y = −1 & y = 1 & y = 3\\
x = −1 & 0.1 & 0.1 & z\\
x = 1 & 0.3 & 0.05 & 0.05\\
\end{tabular}
\end{center}

\begin{align}
\int p(x,y) &= 1 \\
1 &= 0.1 + 0.1 + 0.3 + 0.05 + 0.05 + z = 0.6 + z \\
z &= 1 - 0.6 = 0.4
\end{align}

Skrevet i R:
\begin{verbatim}
p <- matrix(c(0.1,0.1,z,0.3,0.05,0.05), 2, 3, TRUE)
z <- 1 - (0.1 + 0.1 + 0.3 + 0.05 + 0.05)
\end{verbatim}
\subsection*{b}
\label{sec:orge281eb0}
Beregn den forventede gevinst

\begin{align}
E[X+Y] &= E[X] + E[Y] \\
E[X] &= -1*0.6 + 1*0.4 = -0.2 \\
E[Y] &= -1*0.4 + 1*0.15 + 3*0.45 = 1.1 \\
E[X+Y] &= 1.1 - 0.2 = 0.9
\end{align}

Skrevet i R (fortsat):

\begin{verbatim}
x <- -sum(p[1,]) + sum(p[2,])
y <- -sum(p[,1]) + sum(p[,2]) + 3*sum(p[,3])
E <- x + y # Forventede gevinst
\end{verbatim}
\subsection*{c}
\label{sec:org30d2f7e}
beregn kovariansen mellem X og Y. Er X og Y uafhængige?

Skrevet i R (fortsat):
\begin{verbatim}
xy <- c(z, 0.3+0.1, 0.15, 0.05)
-3*xy[1]+ (-1)*xy[2] +xy[3]+3*xy[4] #-> E(xy) = -1.3
-1.3-(mean(x)*mean(y)) #->              Cov(x,y)= -1.08
\end{verbatim}

Hvilket betyder at de er afhængige af hinanden, da cov /= 0 og at de 2 bevæger sig modsat af hinanden

\section*{3}
\label{sec:org1a52b58}
\subsection*{a}
\label{sec:org610609b}
Vi ved at hvis \(A\) har et udfald som er givet,
har den altid en sandsynlighed \(1\) betinget at det er udfaldet.

\begin{center}
\begin{tabular}{lrr}
CPT & A=1 & A=0\\
P(T=1 givet A) & 0.998 & a\\
P(T=0 givet A) & b & 0.993\\
sum & 1 & 1\\
\end{tabular}
\end{center}

Derfor må \(a = 1-0.993\) og \(b=1-0.998\).

\begin{center}
\begin{tabular}{lrr}
CPT & A=1 & A=0\\
P(T=1 givet A) & 0.998 & 0.007\\
P(T=0 givet A) & 0.002 & 0.993\\
sum & 1 & 1\\
\end{tabular}
\end{center}

Sandsynligheden for at have en peanut allergi er \(P(A = 1) = 0.01\), 1\%,
samt \(P(A = 0) = 0.99\), 99\% for ikke at have allergien.

Da
\[P(A|B) = \frac{P(A \union B)}{P(B)}\]
\[P(A \union B) = P(A|B)\cdot P(B)\]

kan en \textbf{pmf} beregnes.

\begin{center}
\begin{tabular}{lrrr}
PMF & A=1 & A=0 & sum\\
T=1 & 0.00998 & 0.00693 & 0.01691\\
T=0 & 0.00002 & 0.98307 & 0.98309\\
sum & 0.01 & 0.99 & 1\\
\end{tabular}
\end{center}

Summen af rækkerne er så hhv. \(P(T=1)\) og \(P(T=0)\).
\subsection*{b}
\label{sec:org874a91e}
Sandsynligheden for ikke at have allergien givet en negativ test er:
\[P(A=0|T=0)=\frac{P(A=0,T=0)}{P(T=0)}=\frac{0.98307}{0.98309}= 0.9999797\]
\subsection*{c}
\label{sec:org5db15a6}
Sandsynligheden for at have allergien givet en positiv test er:
\[P(A=1|T=1)=\frac{P(A=1,T=1)}{P(T=1)}=\frac{0.00998}{0.01}= 0.998\]

\section*{4}
\label{sec:orgacc3a3f}
\subsection*{a}
\label{sec:org673380d}
\begin{equation}
p_i(x) =
\begin{cases}
\frac{1}{365} & \quad if \quad x \in \{1,\cdots,365\}\\
0 & \quad if \quad x \notin \{1,\cdots,365\}
\end{cases}
\end{equation}
\subsection*{b}
\label{sec:orga0002ee}
\texttt{ER DET GODT NOK FORKLARET?} \\
Da \(x_1,x_2,\cdots,x_n\) er disjunkte kan \(P(x_1,x_2,\cdots,x_n)\) udtrykkes som produktet af sandsyndligheden for hvert udfald. \textbf{mangler citation}.
\begin{align}
P(x_1,x_2,\cdots,x_n) &= P(X_i=x_1) \cdot P(X_i=x_2) \cdots P(X_i=x_n)\\
&= p_i(X_i=x_1) \cdot p_i(X_i=x_2) \cdots p_i(X_i=x_n)\\
p_i(x) &=
\begin{cases}
\frac{1}{365} & \quad if \quad x \in \{1,\cdots,365\}\\
0 & \quad ellers
\end{cases} \\
P(x_1,x_2,\cdots,x_n) &=
\begin{cases}
\frac{1}{365^n} & \quad hvis \quad (x_1,x_2,\cdots,x_n) \in \{1,\cdots,365\}\\
0 & \quad ellers
\end{cases}
\end{align}
\subsection*{c}
\label{sec:orgb4abdde}
\begin{align}
\forall x, \quad p(x) &\ge 0 \\
\sum_x p(x) &= 1 \\
\end{align}

\(p_2(x)\) er et produkt af to muglige faktorer \(0\) og \(\frac{1}{365}\).
\[0 \ge 0 \quad \frac{1}{365} \ge 0\]

\(x_1\) kan antage 365 udfald, \(x_2\) 365, og samme for resten op til \(x_n\).
Derfor må pmf; \(P(x_1,x_2,\cdots,x_n)\) have
\(365 \cdot \stackrel{n}{\cdots} \cdot 365 = 365^n\)
udfald.
\[
\sum_{(x_1,x_2,\cdots,x_n) \in \{1,\cdots,365\}} p(x_1,x_2,\cdots,x_n) =
\sum_x \frac{1}{365^n} = 365^n \cdot \frac{1}{365^n} = 1 \\
\]
\subsection*{d}
\label{sec:org14a6f6d}
Sandsynligheden for nogen i en gruppe, dvs. 2 eller mere, har fødseldag på samme dag er det omvendte af at ingen i gruppen har fødseldag på samme dag, dvs. 1 eller mindre.
\begin{align}
B&=\text{"$\ge$ 2 har fødselsdag på samme dag"}\\
B^C&=\text{"$\le$ 1 har fødselsdag på samme dag"} = \text{"Alle har en unik fødselsdag"}
\end{align}

Ethvert udfald vil ligge i enten \(B\) eller \(B^C\), \texttt{ER DET TRIVIELT AT B, BC ER DISJUNKTE?}
\[P(x_1,x_2,\cdots,x_n \in B \cup B^C) = 1\]
\[P(x_1,x_2,\cdots,x_n \in B) = 1 - P(x_1,x_2,\cdots,x_n \in B^C)\]

Mængden af dage hvor man kan have en unik fødselsdag er 365 hvis der blot er én person.
Hvis der er to personer er der 365 dage for den første og 364 for den anden,
da den anden person ikke kan have fødselsdag på samme dag som den første.
Det giver \(365 \cdot 364\) kombinationer.
Det gælder generelt at antalet af kombinationer for \(n\) personer er:
\[365 \cdot (365 - 1) \cdots (365-n+1)\]
Eller omskrevet:
\[\frac{365!}{(365-n)!}\]
Vi kender sandsyndligheden for et specifikt udfald for \(n\) personer,
ved \$P(x\textsubscript{1,x}\textsubscript{2,\(\cdots{}\),x}\textsubscript{n}).
Antages det at alle har rigtige fødselsdage,
kan sandsynligheden for at \(n\) personer alle har unikke fødselsdage ift. hindanden.
\[P(x_1,x_2,\cdots,x_n \in B^C) = \frac{365!}{(365-n)!\cdot 365^n}\]
Dvs.
\[P(x_1,x_2,\cdots,x_n \in B) = 1 - P(x_1,x_2,\cdots,x_n \in B^C)\]
\[P(x_1,x_2,\cdots,x_n \in B) = 1 - \frac{365!}{(365-n)!\cdot 365^n}\]

Skrevet i R (muligvis med forstærkede afrundingsfejl):
\begin{verbatim}
p = \(n) 1 - prod(c((365-n):365)/365)
p(10) # 0.1411414
p(20) # 0.4436883
p(50) # 0.974432
\end{verbatim}
\end{document}
