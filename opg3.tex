% Created 2021-12-11 Sat 14:50
% Intended LaTeX compiler: pdflatex
\documentclass[11pt]{article}
\usepackage[utf8]{inputenc}
\usepackage[T1]{fontenc}
\usepackage{graphicx}
\usepackage{grffile}
\usepackage{longtable}
\usepackage{wrapfig}
\usepackage{rotating}
\usepackage[normalem]{ulem}
\usepackage{amsmath}
\usepackage{textcomp}
\usepackage{amssymb}
\usepackage{capt-of}
\usepackage{hyperref}
\author{Ali Kadum Hassan, Frederik Henriques Altmann, Gustav Emil Mark-Hansen}
\date{\today}
\title{Opg3}
\hypersetup{
 pdfauthor={Ali Kadum Hassan, Frederik Henriques Altmann, Gustav Emil Mark-Hansen},
 pdftitle={Opg3},
 pdfkeywords={},
 pdfsubject={},
 pdfcreator={Emacs 27.2 (Org mode 9.5)}, 
 pdflang={English}}
\begin{document}

\maketitle

\section{OPGAVE 1}
\label{sec:orge3fbc54}

\subsection{a}
\label{sec:org910bea3}
Chancen for at få X kast der lander på krone i træk, er:
\((\frac{1}{2})^X\)

Skrevet i R:
\begin{verbatim}
pmf = \(throws) 0.5^throws
\end{verbatim}
\subsection{b}
\label{sec:orgf2dfbaf}
Profiten \(f(X)\) er givet ved \(f(a, X) = a2^X - a\)

Skrevet i R:
\begin{verbatim}
profit = \(buyin, throws) buyin * 2^(throws) - buyin
\end{verbatim}
\subsection{c}
\label{sec:org54263c6}
\begin{align}
E(X) = \sum_{n=1}^\infty |f(b,n)|p(n) = \sum_{n=1}^\infty b \\
|f(x)|p(x) = b2^{x+1}0.5^x = b \\
\sum_{n=1}^\infty |f(b,n)|p(n) = \sum_{n=1}^\infty b = \infty
\end{align}

Da \(E(X)\) divergerer mod positiv uendelig har spillet ikke noget forventet udfald.
\section{2}
\label{sec:orga1b3501}
\subsection{a}
\label{sec:org9707988}
Da hele pmf \(p\), bortset fra \(p(x=-1,y=3)\) er kendt kan \(z\) beregnes ved at isolere.

\begin{center}
\begin{tabular}{lrrl}
p(x, y) & y = −1 & y = 1 & y = 3\\
x = −1 & 0.1 & 0.1 & z\\
x = 1 & 0.3 & 0.05 & 0.05\\
\end{tabular}
\end{center}

\begin{align}
\int p(x,y) &= 1 \\
1 &= 0.1 + 0.1 + 0.3 + 0.05 + 0.05 + z = 0.6 + z \\
z &= 1 - 0.6 = 0.4
\end{align}

Skrevet i R:
\begin{verbatim}
p <- matrix(c(0.1,0.1,z,0.3,0.05,0.05), 2, 3, TRUE)
z <- 1 - (0.1 + 0.1 + 0.3 + 0.05 + 0.05)
\end{verbatim}
\subsection{b}
\label{sec:org7f51812}
Beregn den forventede gevinst

\begin{align}
E[X+Y] &= E[X] + E[Y] \\
E[X] &= -1*0.6 + 1*0.4 = -0.2 \\
E[Y] &= -1*0.4 + 1*0.15 + 3*0.45 = 1.1 \\
E[X+Y] &= 1.1 - 0.2 = 0.9
\end{align}

Skrevet i R (fortsat):

\begin{verbatim}
x <- -sum(p[1,]) + sum(p[2,])
y <- -sum(p[,1]) + sum(p[,2]) + 3*sum(p[,3])
E <- x + y # Forventede gevinst
\end{verbatim}
\subsection{c}
\label{sec:org2e1cd51}
beregn kovariansen mellem X og Y. Er X og Y uafhængige?

Skrevet i R (fortsat):
\begin{verbatim}
xy <- c(z, 0.3+0.1, 0.15, 0.05)
-3*xy[1]+ (-1)*xy[2] +xy[3]+3*xy[4] #-> E(xy) = -1.3
-1.3-(mean(x)*mean(y)) #->              Cov(x,y)= -1.08
\end{verbatim}

Hvilket betyder at de er afhængige af hinanden, da cov /= 0 og at de 2 bevæger sig modsat af hinanden

\section{3}
\label{sec:orgcade4cc}
\subsection{a}
\label{sec:orgddb580a}
Vi ved at hvis \(A\) har et udfald som er givet,
har den altid en sandsynlighed \(1\) betinget at det er udfaldet.

\begin{center}
\begin{tabular}{lrr}
CPT & A=1 & A=0\\
P(T=1 givet A) & 0.998 & a\\
P(T=0 givet A) & b & 0.993\\
sum & 1 & 1\\
\end{tabular}
\end{center}

Derfor må \(a = 1-0.993\) og \(b=1-0.998\).

\begin{center}
\begin{tabular}{lrr}
CPT & A=1 & A=0\\
P(T=1 givet A) & 0.998 & 0.007\\
P(T=0 givet A) & 0.002 & 0.993\\
sum & 1 & 1\\
\end{tabular}
\end{center}

Sandsynligheden for at have en peanut allergi er \(P(A = 1) = 0.01\), 1\%,
samt \(P(A = 0) = 0.99\), 99\% for ikke at have allergien.

Da
\[P(A|B) = \frac{P(A \union B)}{P(B)}\]
\[P(A \union B) = P(A|B)\cdot P(B)\]

kan en \textbf{pmf} beregnes.

\begin{center}
\begin{tabular}{lrrr}
PMF & A=1 & A=0 & sum\\
T=1 & 0.00998 & 0.00693 & 0.01691\\
T=0 & 0.00002 & 0.98307 & 0.98309\\
sum & 0.01 & 0.99 & 1\\
\end{tabular}
\end{center}

Summen af rækkerne er så hhv. \(P(T=1)\) og \(P(T=0)\).
\subsection{b}
\label{sec:org1986b85}
Sandsynligheden for ikke at have allergien givet en negativ test er:
\[P(A=0|T=0)=\frac{P(A=0,T=0)}{P(T=0)}=\frac{0.98307}{0.98309}= 0.9999797\]
\subsection{c}
\label{sec:org7e349ac}
Sandsynligheden for at have allergien givet en positiv test er:
\[P(A=1|T=1)=\frac{P(A=1,T=1)}{P(T=1)}=\frac{0.00998}{0.01}= 0.998\]

\section{4}
\label{sec:org962979d}
\subsection{a}
\label{sec:orgd035d0f}
\begin{equation}
p_1(x) =
\begin{cases}
\frac{1}{365} & \quad if \quad x \in \{1,\cdots,365\}\\
0 & \quad if \quad x \in \mathbb{R} \setminus \{1,\cdots,365\}
\end{cases}
\end{equation}
\subsection{b}
\label{sec:org9002a13}
Sandsynligheden for et givet udfald er stadig uniformt for enhver vektor \(V = (X_1,\cdots,X_n)\).
Derfor er sandsyndligheden for et specifikt udfald for alle elementer i vektoren produktet af de individuelle elementer.
\[
p_2(V) = \prod_{i=1}^n p_1(x_i)
\]
\subsection{c}
\label{sec:org89602e0}
\begin{align}
\forall x, \quad p(x) &\ge 0 \\
\sum_x p(x) &= 1 \\
\end{align}

\(p_2(x)\) er et produkt af to muglige faktorer \(0\) og \(\frac{1}{365}\),
derfor er \(Im(p_2) = [\frac{1}{365};0]\).
\[0 \ge 0 \quad \frac{1}{365} \ge 0\]

Udfaldsrummet er af størrelse \(u^d\) hvor \(d\) antal elementer i vektoren og \(u\) er antallet af udfald per element.
Da udfaldrummet er uniformt må et udfald give at \(p(V) = \frac{1}{u^d}\).

\begin{align}
X_i \in \{1,\cdots,365\} \implies p_2(V) &= \prod_{i=1}^n \frac{1}{365} \\
&= \frac{1}{365^n} \\
&= \frac{1}{u^d}
\end{align}
\subsection{d}
\label{sec:orgee39b1c}
Sandsynligheden for nogen i en gruppe har fødseldag på samme dag er det omvendte (\(1-p\)) af at ingen i gruppen har fødseldag på samme dag.
Denne betingede sandsyndlighed er \(1\) for \(n=0\) og \(1\frac{364}{365}\) for \(n=1\),
fordi den første fødseldag fjerner en dag fra udfaldsrummet hvor fødseldagene ikke kolliderer.
Generelt er sekvensen \(\frac{365}{365}\frac{364}{365}\cdots\frac{365-n}{365}\).
Dette kan omskrives til \(\frac{1}{365^n}\frac{365!}{(365-n)!}\).
Dvs. \(p(n) = 1- \frac{1}{365^n}\frac{365!}{(365-n)!}\).

Skrevet i R (muligvis med forstærkede afrundingsfejl):
\begin{verbatim}
p = \(n) 1 - prod(c((365-n):365)/365)
p(10) # 0.1411414
p(20) # 0.4436883
p(50) # 0.974432
\end{verbatim}
\end{document}
